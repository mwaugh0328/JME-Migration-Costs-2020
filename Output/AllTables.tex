\documentclass[11pt]{article}% Your documentclass
\usepackage{verbatim}
\usepackage[margin=1.5cm]{geometry}
\usepackage{dcolumn}
\usepackage{comment}
\usepackage{fancyhdr}
\usepackage{graphicx}
% Necessary packages
\usepackage[T1]{fontenc}% Must be loaded for proper fontencoding when using pdfLaTeX
\usepackage[utf8]{inputenx}% For proper input encoding

% Packages for tables
\usepackage{booktabs}% Pretty tables
\usepackage{threeparttable}% For Notes below table
\usepackage[skip=5pt, justification=centering]{caption}
\usepackage{longtable}
\usepackage{pdflscape}
\usepackage{amsmath}
\usepackage{morefloats}
%INCLUDE HYPERREF AT END
\usepackage{hyperref}
\hypersetup{
  colorlinks   = true, %Colours links instead of ugly boxes
  linkcolor    = blue, %Colour of internal links
}
%%NEED THIS TO WORK WITH TABLES%%
      \makeatletter
       \edef\originalbmathcode{%
           \noexpand\mathchardef\noexpand\@tempa\the\mathcode`\(\relax}
       \def\resetMathstrut@{%
         \setbox\z@\hbox{%
           \originalbmathcode
          \def\@tempb##1"##2##3{\the\textfont"##3\char"}%
 \expandafter\@tempb\meaning\@tempa \relax
 }%
 \ht\Mathstrutbox@\ht\z@ \dp\Mathstrutbox@\dp\z@
 }
 \makeatother

%TEXTFONT
 \usepackage{mweights}
 %DEFAULT
%MATHFONT
 %DEFAULT
% *****************************************************************
% siunitx
% *****************************************************************
\newcommand{\sym}[1]{\rlap{#1}} % Thanks to Joseph Wright & David Carlisle

\usepackage{siunitx}
 \sisetup{
 detect-mode,
 group-digits = false,
 input-symbols = ( ) [ ] - +,
 table-align-text-post = false,
 input-signs = ,
 } 

% Character substitution that prints brackets and the minus symbol in text mode. Thanks to David Carlisle
\def\yyy{%
 \bgroup\uccode`\~\expandafter`\string-%
 \uppercase{\egroup\edef~{\noexpand\text{\llap{\textendash}\relax}}}%
 \mathcode\expandafter`\string-"8000 }

\def\xxxl#1{%
\bgroup\uccode`\~\expandafter`\string#1%
\uppercase{\egroup\edef~{\noexpand\text{\noexpand\llap{\string#1}}}}%
\mathcode\expandafter`\string#1"8000 }

\def\xxxr#1{%
\bgroup\uccode`\~\expandafter`\string#1%
\uppercase{\egroup\edef~{\noexpand\text{\noexpand\rlap{\string#1}}}}%
\mathcode\expandafter`\string#1"8000 }

\def\textsymbols{\xxxl[\xxxr]\xxxl(\xxxr)\yyy}
% *****************************************************************
% Estout related things
% *****************************************************************
\let\estinput=\input % define a new input command so that we can still flatten the document

\newcommand{\estwide}[3]{
               \vspace{.75ex}{
                       \textsymbols
                       \begin{tabular*}
                       {\textwidth}{@{\hskip\tabcolsep\extracolsep\fill}l*{#2}{#3}}
                       \toprule
                       \estinput{#1}
                       \bottomrule
                       \addlinespace[0.75ex]
                       \end{tabular*}
                       }
               }       

\newcommand{\estwideland}[3]{
               \vspace{.75ex}{
                       \textsymbols
                       \begin{tabular*}
                       {\linewidth}{@{\hskip\tabcolsep\extracolsep\fill}l*{#2}{#3}}
                       \toprule
                       \estinput{#1}
                       \bottomrule
                       \addlinespace[0.75ex]
                       \end{tabular*}
                       }
               }       

\newcommand{\estauto}[3]{
               \vspace{.75ex}{
                       \textsymbols
                       \begin{tabular}{l*{#2}{#3}}
                       \toprule
                       \estinput{#1}
                       \bottomrule
                       \addlinespace[.75ex]
                       \end{tabular}
                       }
               }

% Allow line breaks with \\ in specialcells
\newcommand{\specialcell}[2][c]{%
    \begin{tabular}[#1]{@{}c@{}}#2\end{tabular}
}

% *****************************************************************
% Custom subcaptions
% *****************************************************************
% Note/Source/Text after Tables
% The new approach using threeparttables to generate notes that are the exact width of the table.
\newcommand{\Figtext}[1]{%
       \begin{tablenotes}[para,flushleft]
       {
       #1
       }
       \end{tablenotes}
       
       }
% *****************************************************************
% END PREAMBLE
% *****************************************************************

\begin{document}
\begin{center}
\smallskip\begin{large}Characteristics of six panel datasets\end{large}\\
\smallskip
\begin{tabular}{lcccccc}
\hline \noalign{\smallskip} & Waves & Communities & Individuals & Observations & P(Rural) & P(Migrate R\-U)\\
\noalign{\smallskip}\hline \noalign{\smallskip}China & 4 & 176 & 50,965 & 143,923 & .54 & .026\\
Ghana & 2 & 334 & 7,633 & 15,027 & .64 & .01\\
Indonesia & 5 & 296 & 48,184 & 131,803 & .63 & .023\\
Malawi & 3 & 204 & 13,969 & 38,165 & .72 & .008\\
South Africa & 5 & 400 & 38,430 & 104,090 & .49 & .025\\
Tanzania & 3 & 409 & 15,887 & 35,103 & .62 & .024\\
\noalign{\smallskip}\hline\end{tabular}\\
\begin{footnotesize}Note: Columns 1–4 list the number of survey waves\end{footnotesize}\\
\begin{footnotesize}the number of communities (i.e. enumeration areas) surveyed\end{footnotesize}\\
\begin{footnotesize}the number of individuals surveyed\end{footnotesize}\\
\begin{footnotesize}and the total number of observations\end{footnotesize}\\
\begin{footnotesize}for each country. Column 5 lists the fraction of adults living in a rural location in wave 1. Column 6 presents the annualized rural-urban migration rate for adults in wave 1.        \end{footnotesize}\\
\smallskip
\end{center}

\begin{table}[htbp]\centering
\def\sym#1{\ifmmode^{#1}\else\(^{#1}\)\fi}
\caption{Observational Returns to Migration in Six Developing Countries.}
\begin{tabular}{l*{4}{c}}
\toprule
                    &\multicolumn{1}{c}{(1)}&\multicolumn{1}{c}{(2)}&\multicolumn{1}{c}{(3)}&\multicolumn{1}{c}{(4)}\\
                    &\multicolumn{1}{c}{} &\multicolumn{1}{c}{} &\multicolumn{1}{c}{} &\multicolumn{1}{c}{} \\
\midrule
China               &       0.545\sym{***}&       0.161\sym{***}&       0.012         &       0.226\sym{***}\\
                    &     (0.005)         &     (0.028)         &     (0.064)         &     (0.031)         \\
\addlinespace
Ghana               &       0.410\sym{***}&       0.148         &      -0.173         &       0.339\sym{**} \\
                    &     (0.013)         &     (0.122)         &     (0.220)         &     (0.148)         \\
\addlinespace
Indonesia           &       0.625\sym{***}&       0.145\sym{***}&       0.039         &       0.167\sym{***}\\
                    &     (0.009)         &     (0.019)         &     (0.031)         &     (0.029)         \\
\addlinespace
Malawi              &       0.520\sym{***}&       0.048         &      -0.350\sym{***}&       0.189         \\
                    &     (0.012)         &     (0.089)         &     (0.123)         &     (0.134)         \\
\addlinespace
South Africa        &       0.737\sym{***}&       0.212\sym{***}&       0.028         &       0.291\sym{***}\\
                    &     (0.006)         &     (0.022)         &     (0.044)         &     (0.026)         \\
\addlinespace
Tanzania            &       0.666\sym{***}&       0.112\sym{***}&       0.101\sym{**} &       0.213\sym{***}\\
                    &     (0.032)         &     (0.030)         &     (0.045)         &     (0.043)         \\
\midrule
Individual\_FE       &          No         &         Yes         &         Yes         &         Yes         \\
Year\_FE             &          No         &         Yes         &         Yes         &         Yes         \\
Sample              &        Full         &        Full         & Start Urban         & Start Rural         \\
\bottomrule
\multicolumn{5}{l}{\footnotesize Note: This table presents the estimated coefficients of urban dummy}\\
\multicolumn{5}{l}{\footnotesize variables from regressions of log consumption per adult on urban dummies}\\
\multicolumn{5}{l}{\footnotesize and other covariates in the six countries. Column (1) presents}\\
\multicolumn{5}{l}{\footnotesize the cross-sectional estimates, with no other controls. Column (2) adds}\\
\multicolumn{5}{l}{\footnotesize year and individual fixed effects, plus quadratic controls for age and}\\
\multicolumn{5}{l}{\footnotesize household size. Column (3) has year and individual fixed effects, plus}\\
\multicolumn{5}{l}{\footnotesize quadratic controls for age and household size, and restricts the sample to}\\
\multicolumn{5}{l}{\footnotesize only those starting in an urban location. Column (4) is the same model}\\
\multicolumn{5}{l}{\footnotesize as in column (3), but restricts the sample to only those starting from a}\\
\multicolumn{5}{l}{\footnotesize rural location. Robust standard errors, clustered at the level of the wave}\\
\multicolumn{5}{l}{\footnotesize 1 household, are in parenthesis. $\sym{*} p<.1, \sym{**}p<.05, \sym{***}p<.01$}\\
\end{tabular}
\end{table}

\begin{table}[htbp]\centering
\def\sym#1{\ifmmode^{#1}\else\(^{#1}\)\fi}
\caption{Observational Returns to Migration: Income Measures.}
\begin{tabular}{l*{4}{c}}
\toprule
                    &\multicolumn{1}{c}{(1)}&\multicolumn{1}{c}{(2)}&\multicolumn{1}{c}{(3)}&\multicolumn{1}{c}{(4)}\\
                    &\multicolumn{1}{c}{} &\multicolumn{1}{c}{} &\multicolumn{1}{c}{} &\multicolumn{1}{c}{} \\
\midrule
Indonesia           &       0.623\sym{***}&       0.169\sym{***}&       0.014         &       0.100\sym{**} \\
                    &     (0.011)         &     (0.028)         &     (0.042)         &     (0.042)         \\
\addlinespace
South Africa        &       0.649\sym{***}&       0.249\sym{***}&       0.035         &       0.289\sym{***}\\
                    &     (0.006)         &     (0.023)         &     (0.045)         &     (0.029)         \\
\midrule
hours\_control       &                     &                     &                     &                     \\
Individual\_FE       &          No         &         Yes         &         Yes         &         Yes         \\
Year\_FE             &          No         &         Yes         &         Yes         &         Yes         \\
Sample              &        Full         &        Full         & Start Urban         & Start Rural         \\
\bottomrule
\multicolumn{5}{l}{\footnotesize Note: This table presents the estimated coefficients of urban dummy}\\
\multicolumn{5}{l}{\footnotesize variables from regressions of log household income per adult on urban}\\
\multicolumn{5}{l}{\footnotesize dummies and other covariates in the six countries. Column (1) presents}\\
\multicolumn{5}{l}{\footnotesize the cross-sectional estimates, with no other controls. Column (2) adds}\\
\multicolumn{5}{l}{\footnotesize year and individual fixed effects, plus quadratic controls for age and}\\
\multicolumn{5}{l}{\footnotesize household size. Column (3) has year and individual fixed effects, plus}\\
\multicolumn{5}{l}{\footnotesize quadratic controls for age and household size, and restricts the sample to}\\
\multicolumn{5}{l}{\footnotesize only those starting in an urban location. Column (4) is the same model}\\
\multicolumn{5}{l}{\footnotesize as in column (3), but restricts the sample to only those starting from a}\\
\multicolumn{5}{l}{\footnotesize rural location. Robust standard errors, clustered at the level of the wave}\\
\multicolumn{5}{l}{\footnotesize 1 household, are in parenthesis. $\sym{*} p<.1, \sym{**}p<.05, \sym{***}p<.01$}\\
\end{tabular}
\end{table}

\begin{table}[htbp]\centering
\def\sym#1{\ifmmode^{#1}\else\(^{#1}\)\fi}
\caption{Results for Indonesia Under Alternative Specifications.}
\begin{tabular}{l*{4}{c}}
\toprule
                    &\multicolumn{1}{c}{(1)}&\multicolumn{1}{c}{(2)}&\multicolumn{1}{c}{(3)}&\multicolumn{1}{c}{(4)}\\
                    &\multicolumn{1}{c}{Cross-Section}&\multicolumn{1}{c}{Panel}&\multicolumn{1}{c}{Rural-Only}&\multicolumn{1}{c}{Work Moves}\\
\midrule
City, Ind. Earnings &       0.638\sym{***}&       0.101\sym{***}&       0.081         &       0.016         \\
                    &     (0.019)         &     (0.038)         &     (0.066)         &     (0.134)         \\
\addlinespace
City, HH. Earnings  &       0.623\sym{***}&       0.169\sym{***}&       0.100\sym{**} &       0.229\sym{***}\\
                    &     (0.011)         &     (0.028)         &     (0.042)         &     (0.086)         \\
\addlinespace
City, HH.           &       0.625\sym{***}&       0.145\sym{***}&       0.167\sym{***}&       0.263\sym{***}\\
Consumption         &     (0.009)         &     (0.019)         &     (0.029)         &     (0.056)         \\
\addlinespace
City + small towns, &       0.739\sym{***}&       0.065\sym{**} &       0.044         &       0.071         \\
Ind. Earnings       &     (0.014)         &     (0.030)         &     (0.034)         &     (0.073)         \\
\addlinespace
City + small towns, &       0.590\sym{***}&       0.116\sym{***}&       0.100\sym{***}&       0.178\sym{***}\\
HH. Earnings        &     (0.008)         &     (0.021)         &     (0.023)         &     (0.055)         \\
\addlinespace
City + small towns, &       0.597\sym{***}&       0.125\sym{***}&       0.111\sym{***}&       0.156\sym{***}\\
HH. Consumption     &     (0.006)         &     (0.014)         &     (0.015)         &     (0.036)         \\
\midrule
Individual\_FE       &          No         &         Yes         &         Yes         &         Yes         \\
Year\_FE             &         Yes         &         Yes         &         Yes         &         Yes         \\
Initial\_Location    &         All         &         All         &       Rural         &       Rural         \\
Move\_Reason         &         Any         &         Any         &         Any         &        Work         \\
Local\_Mig\_Rate      &         All         &         All         &         All         &         All         \\
\bottomrule
\multicolumn{5}{l}{\footnotesize Standard errors in parentheses}\\
\multicolumn{5}{l}{\footnotesize \sym{*} \(p<.1\), \sym{**} \(p<.05\), \sym{***} \(p<.01\)}\\
\end{tabular}
\end{table}

\begin{table}[htbp]\centering
\def\sym#1{\ifmmode^{#1}\else\(^{#1}\)\fi}
\caption{Observational Returns to Migration by Region.}
\begin{tabular}{l*{3}{c}}
\toprule
                    &\multicolumn{1}{c}{(1)}&\multicolumn{1}{c}{(2)}&\multicolumn{1}{c}{(3)}\\
                    &\multicolumn{1}{c}{log(Cons. PA)}&\multicolumn{1}{c}{log(Cons. PA)}&\multicolumn{1}{c}{p: C1 = C2}\\
\midrule
China               &       0.108\sym{**} &       0.233\sym{***}&       0.043         \\
                    &     (0.046)         &     (0.043)         &                     \\
\addlinespace
Ghana               &       0.240         &       0.428\sym{*}  &       0.507         \\
                    &     (0.174)         &     (0.228)         &                     \\
\addlinespace
Indonesia           &       0.074\sym{*}  &       0.260\sym{***}&       0.001         \\
                    &     (0.040)         &     (0.041)         &                     \\
\addlinespace
Malawi              &       0.287         &       0.121         &       0.532         \\
                    &     (0.199)         &     (0.179)         &                     \\
\addlinespace
South Africa        &       0.303\sym{***}&       0.277\sym{***}&       0.616         \\
                    &     (0.039)         &     (0.037)         &                     \\
\addlinespace
Tanzania            &       0.071         &       0.357\sym{***}&       0.000         \\
                    &     (0.052)         &     (0.061)         &                     \\
\midrule
Individual\_FE       &         Yes         &         Yes         &                     \\
Year\_FE             &         Yes         &         Yes         &                     \\
Sample              &High Migration         &Low Migration         &                     \\
\bottomrule
\multicolumn{4}{l}{\footnotesize Note: This table presents the estimated coefficients of urban dummy variables}\\
\multicolumn{4}{l}{\footnotesize from regressions of log consumption per adult on urban dummies. Specifications}\\
\multicolumn{4}{l}{\footnotesize are as in Table 3 , Column (4), with year and individual fixed effects,}\\
\multicolumn{4}{l}{\footnotesize plus quadratic controls in age and household size, and restricting the sample}\\
\multicolumn{4}{l}{\footnotesize to only those starting from a rural location. The sample is divided by}\\
\multicolumn{4}{l}{\footnotesize the rural-urban migration rate in the origin community, so that there are an}\\
\multicolumn{4}{l}{\footnotesize equal number of rural-urban migrants in each group. Column (2) restricts the}\\
\multicolumn{4}{l}{\footnotesize sample to include households from enumeration areas with rural-urban migration}\\
\multicolumn{4}{l}{\footnotesize rates above the median rate for rural-urban migrants. Column (2) re-}\\
\multicolumn{4}{l}{\footnotesize stricts the sample to include households from enumeration areas with rural-}\\
\multicolumn{4}{l}{\footnotesize urban migration rates below the median rate for rural-urban migrants. Column}\\
\multicolumn{4}{l}{\footnotesize (3) reports the p -value of the difference between the estimates in Column}\\
\multicolumn{4}{l}{\footnotesize (1) and Column (2). Robust standard errors, clustered at the level of the}\\
\multicolumn{4}{l}{\footnotesize wave 1 household, are in parenthesis. $\sym{*} p<.1, \sym{**}p<.05, \sym{***}p<.01$}\\
\end{tabular}
\end{table}

\begin{tabular}{lccc} \hline
 & (1) & (2) & (3) \\
VARIABLES & Observational & Experimental & Difference \\ \hline
 &  &  &  \\
Seasonally Migrated & 0.092* & 0.357** &  \\
 & (0.053) & (0.156) &  \\
Difference in Returns &  &  & 0.265*** \\
 &  &  & (0.095) \\
Constant & 6.739*** & 6.652*** &  \\
 & (0.022) & (0.084) &  \\
 &  &  &  \\
Observations & 1,194 & 1,867 & 1,867 \\
R-squared & 0.699 &  &  \\
Individual FE & Yes & Yes &  \\
 Year FE & Yes & Yes &  \\ \hline
\end{tabular}

\end{document}